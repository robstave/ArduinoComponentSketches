% Tikz-timing example of PCI read operation.
%
% Author: Nathan Typanski <http://nathantypanski.com>
% License: Creative Commons Attribution 4.0 International License
%          <http://creativecommons.org/licenses/by/4.0/>

%% run on www.overleaf.com

\documentclass{standalone}

\usepackage{xparse} % NewDocumentCommand, IfValueTF, IFBooleanTF
\usepackage{tikz-timing}[2014/10/29]
\usepackage{capt-of}
\usetikztiminglibrary[rising arrows]{clockarrows}

% Reference a bus.
%
% Usage:
%
%     \busref[3::0]{C/BE}    ->   C/BE[3::0]
%     \busref*{AD}           ->   AD#
%     \busref*[3::0]{C/BE}   ->   C/BE[3::0]#
%
\NewDocumentCommand{\busref}{som}{\texttt{%
#3%
\IfValueTF{#2}{[#2]}{}%
\IfBooleanTF{#1}{\#}{}%
}}

\begin{document}

 
\begin{tikztimingtable}[%
    timing/dslope=0.1,
    timing/.style={x=5ex,y=2ex},
    x=5ex,
    timing/rowdist=3ex,
    timing/name/.style={font=\sffamily\scriptsize}
]
\busref{CLK in}         & lhlhlhllhhlhlhllhlhhh \\
\busref*{Trigger in}    & 1u H 8L \\
\busref*{Gate}          & 1u 5H 3L  \\
\busref*{Out}           & 1u hlhlhllhhllllllll \\
\extracode
\begin{pgfonlayer}{background}
\begin{scope}[semitransparent ,semithick]
\vertlines[darkgray,dotted]{0.5,1.5 ,...,8.0}

\end{scope}

 
\end{pgfonlayer}


\end{tikztimingtable}

 

\end{document}

