% Tikz-timing example of PCI read operation.
%
% Author: Nathan Typanski <http://nathantypanski.com>
% License: Creative Commons Attribution 4.0 International License
%          <http://creativecommons.org/licenses/by/4.0/>

%% run on www.overleaf.com

\documentclass{article}
\usepackage{xparse} % NewDocumentCommand, IfValueTF, IFBooleanTF
\usepackage{tikz-timing}[2014/10/29]
\usepackage{tikz}
\usetikzlibrary{fit} % positioning, calc libraries may also be useful
\usepackage{capt-of}
\usepackage{changepage} 
\usetikztiminglibrary[rising arrows]{clockarrows}
% Reference a bus.
%
% Usage:
%
%     \busref[3::0]{C/BE}    ->   C/BE[3::0]
%     \busref*{AD}           ->   AD#
%     \busref*[3::0]{C/BE}   ->   C/BE[3::0]#
%
\NewDocumentCommand{\busref}{som}{\texttt{%
#3%
\IfValueTF{#2}{[#2]}{}%
\IfBooleanTF{#1}{\#}{}%
}}

\begin{document}
\begin{center}
    \huge\textbf{ACS-85-0610}
    
    \large\textbf{Clock Counter (Quadrature)/seq}

\end{center}

\tikzstyle{icdev}=[draw, text width=10em, minimum height=8em]

\begin{center}
\begin{tikztimingtable}[%
    timing/dslope=0.1,
    timing/.style={x=5ex,y=2ex},
    x=5ex,
    timing/rowdist=3ex,
    timing/name/.style={font=\sffamily\scriptsize}
]
\busref{CLK in}         & lhlhlhlhl \\
\busref*{0}        & lhhllhhll \\
\busref*{1}         & llhhllhhl \\
\busref*{2}         & lllhhllhh \\
\busref*{2}         & lhllhhllh \\
\extracode
\begin{pgfonlayer}{background}
\begin{scope}[semitransparent ,semithick]
\vertlines[darkgray,dotted]{0.5,1.5 ,...,11.0}

\end{scope}
\end{pgfonlayer}
\end{tikztimingtable}
 
  
 
\end{center}

\begin{adjustwidth}{2cm}{2cm}
   \small\textbf{Clock is really a sequence of patterns.  It can be adjusted as needed}
     \end{adjustwidth} 
\end{document}

